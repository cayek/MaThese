%% common 
\newcommand{\matr}[1]{\mathbf{#1}} %% \matr{}
\newcommand{\Y}{\matr{Y}} %% Output matrix
\newcommand{\Yrow}{n} %% Y row dim
\newcommand{\Ycol}{p} %% Y col dim
\newcommand{\Id}{\matr{Id}} %% \Id
\newcommand\norm[1]{\left\lVert#1\right\rVert} %% \norm
\newcommand{\pvalue}{$p\text{-valeur}$ } %% \pvalue
\newcommand{\qvalue}{$q\text{-valeur}$ } %% \pvalue
\newcommand{\zscore}{$z\text{-score}$ } %% zscore
\newcommand{\zscores}{$z\text{-scores}$ } %% zscore
\newcommand{\pvalues}{$p\text{-valeurs}$ } %% \pvalue
\newcommand{\celiac}{c\oe{}liaque } %% \pvalue
\newcommand{\Var}{\mathrm{Var}} %% variance
\newcommand{\sign}{\mathrm{sign}} %% signe
\newcommand{\RR}{\mathbb{R}} %% real ensemble
\newcommand{\Normal}{\mathcal{N}} %% Normal distribution
\newcommand{\svd}{\mathrm{svd}} %% la svd
\newcommand{\med}{\mathrm{median}} %% la median
\newcommand{\mad}{\mathrm{MAD}} %% le MAD

%% name of method
\newcommand{\MethodCate}{CATE}

%% name of software
\newcommand{\RpackageCate}{{\tt cate}}


%% lattent factor model
\newcommand{\K}{K} %% number of latent factor
\newcommand{\U}{\matr{U}} %% factor score matrix/ records latent variables
\newcommand{\Ucol}{\K} %% U col dim
\newcommand{\Urow}{\Yrow} %% U row dim
\newcommand{\V}{\matr{V}} %% factor loading matrix
\newcommand{\Vcol}{\K} %% V col dim
\newcommand{\Vrow}{\Ycol} %% V row dim
\newcommand{\X}{\matr{X}} %% primary matrix / record primary variable
\newcommand{\Xcol}{d} %% X col dim
\newcommand{\Xrow}{\Yrow} %% X row dim
\newcommand{\B}{\matr{B}} %% primary effects matrix
\newcommand{\Bcol}{\Xcol} %% B col dim
\newcommand{\Brow}{\Ycol} %% B row dim
\newcommand{\E}{\matr{E}} %% residual error matrix
\newcommand{\W}{\matr{W}} %% approximation matrix with latent factor/
                          %% approximation matrix
\newcommand{\Lnoreg}{\mathcal{L}} %% L 
\newcommand{\Llasso}{\mathcal{L}_{\mathrm{lasso}}} %% L lasso 
\newcommand{\Lridge}{\mathcal{L}_{\mathrm{ridge}}} %% L ridge
\newcommand{\Q}{\matr{Q}} %% svd de X
\newcommand{\D}{\matr{D}_{\lambRidge}} %% matrice utilisé dans l'estimateur ridge
\newcommand{\lambRidge}{\lambda} %% reg devant norme L2
\newcommand{\lambLasso}{\mu} %% reg devant norme L1

%% lfmm equation
\newcommand{\LfmmLridge}{%
\Lridge(\U, \V, \B) =  \frac{1}{2} \norm{\Y - \U \V^{T} - \X \B^T}_{F}^2 +
\frac{\lambRidge}{2} \norm{\B}^{2}_{2}%
}
\newcommand{\LfmmL}{%
\Lnoreg(\U, \V, \B) =  \frac{1}{2} \norm{\Y - \U \V^T - \X \B^T}_{F}^2 %
}
\newcommand{\LfmmLlasso}{%
\Llasso(\W, \B) =  \frac{1}{2} \norm{\Y - \W - \X \B^T}_{F}^2 +
\lambLasso \norm{\B}_{1} + \gamma \norm{\W}_{*}%
}


%% tess3 equation
\newcommand{\G}{\matr{G}} %% matrice des fréquences ancestrales de génotypes
\newcommand{\Q}{\matr{Q}} %% matrice des coéficients de métissage individuels
\newcommand{\DQ}{\Delta_{Q}} %% ensemble des contraintes
\newcommand{\DG}{\Delta_{G}} %% ensemble des contraintes
\newcommand{\W}{\matr{W}} %% matrice de poid
\newcommand{\Laplacienne}{\matr{\Gamma}} %% matrice de poid
\newcommand{\LS}{\mathcal{L}} %% optimim function de tess3
\newcommand{\U}{\matr{U}} %% eigen vectors de la laplacienne
\newcommand{\LapVp}{\matr{\Delta}} %% eigen values de la laplacienne
\newcommand{\Fst}{F_{\rm ST}} %% Fst
